%%% Šablona pro jednoduchý soubor formátu PDF/A, jako treba samostatný abstrakt práce.

\documentclass[12pt]{report}

\usepackage[a4paper, hmargin=1in, vmargin=1in]{geometry}
\usepackage[a-2u]{pdfx}
\usepackage[czech]{babel}
\usepackage[utf8]{inputenc}
\usepackage[T1]{fontenc}
\usepackage{lmodern}
\usepackage{textcomp}

\begin{document}

%% Nezapomeňte upravit abstrakt.xmpdata.

Tato práce se zabývá problémem automatického generovaní lékařských zpráv v českém jazyce na základě vstupních rentgenových snímků hrudníku pomocí hlubokých neuronových sítí. První část se zabývá analýzou problému samotného včetně porovnání existujících řešení z několika společných úhlů pohledu. Za účelem interpretace lékařských snímků v českém jazyce představujeme natrénovaný český GPT-2 model specializovaný na lékařské texty, který vychází z původního předtrénovaného anglického GPT-2 modelu, spolu s jeho vyhodnocením. Ve druhé části je vytvořené české GPT-2 použito pro trénování modelu neuronové sítě pro generování lékařských zpráv. Trénování bylo provedeno na volně dostupných datech spolu s předzpracováním dat a jejich úpravou pro český jazyk. Dále jsou výsledky modelu diskutovány a vyhodnoceny pomocí standardních metrik pro zpracování přirozeného jazyka za účelem určení výkonnosti.
\end{document}
