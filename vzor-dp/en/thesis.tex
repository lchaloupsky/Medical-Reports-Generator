%%% The main file. It contains definitions of basic parameters and includes all other parts.

%% Settings for single-side (simplex) printing
% Margins: left 40mm, right 25mm, top and bottom 25mm
% (but beware, LaTeX adds 1in implicitly)
\documentclass[12pt,a4paper]{report}
\setlength\textwidth{145mm}
\setlength\textheight{247mm}
\setlength\oddsidemargin{15mm}
\setlength\evensidemargin{15mm}
\setlength\topmargin{0mm}
\setlength\headsep{0mm}
\setlength\headheight{0mm}
% \openright makes the following text appear on a right-hand page
\let\openright=\clearpage

%% Settings for two-sided (duplex) printing
% \documentclass[12pt,a4paper,twoside,openright]{report}
% \setlength\textwidth{145mm}
% \setlength\textheight{247mm}
% \setlength\oddsidemargin{14.2mm}
% \setlength\evensidemargin{0mm}
% \setlength\topmargin{0mm}
% \setlength\headsep{0mm}
% \setlength\headheight{0mm}
% \let\openright=\cleardoublepage

%% Generate PDF/A-2u
\usepackage[a-2u]{pdfx}

%% Character encoding: usually latin2, cp1250 or utf8:
\usepackage[utf8]{inputenc}

%% Prefer Latin Modern fonts
\usepackage{lmodern}

%% Further useful packages (included in most LaTeX distributions)
\usepackage{amsmath}        % extensions for typesetting of math
\usepackage{amsfonts}       % math fonts
\usepackage{amsthm}         % theorems, definitions, etc.
\usepackage{bbding}         % various symbols (squares, asterisks, scissors, ...)
\usepackage{bm}             % boldface symbols (\bm)
\usepackage{graphicx}       % embedding of pictures
\usepackage{fancyvrb}       % improved verbatim environment
\usepackage{natbib}         % citation style AUTHOR (YEAR), or AUTHOR [NUMBER]
\usepackage[nottoc]{tocbibind} % makes sure that bibliography and the lists
			    % of figures/tables are included in the table
			    % of contents
\usepackage{dcolumn}        % improved alignment of table columns
\usepackage{booktabs}       % improved horizontal lines in tables
\usepackage{paralist}       % improved enumerate and itemize
\usepackage{xcolor}         % typesetting in color
\usepackage{indentfirst}

%%% Basic information on the thesis
\setcounter{secnumdepth}{4}
\setcounter{tocdepth}{4}

% Thesis title in English (exactly as in the formal assignment)
\def\ThesisTitle{Automatic generation of medical reports from chest X-rays}

% Author of the thesis
\def\ThesisAuthor{Bc. Lukáš Chaloupský}

% Year when the thesis is submitted
\def\YearSubmitted{2022}

% Name of the department or institute, where the work was officially assigned
% (according to the Organizational Structure of MFF UK in English,
% or a full name of a department outside MFF)
\def\Department{Institute of Formal and Applied Linguistics}

% Is it a department (katedra), or an institute (ústav)?
\def\DeptType{Institute}

% Thesis supervisor: name, surname and titles
\def\Supervisor{Mgr. Rudolf Rosa, Ph.D.}

% Supervisor's department (again according to Organizational structure of MFF)
\def\SupervisorsDepartment{Institute of Formal and Applied Linguistics}

% Study programme and specialization
\def\StudyProgramme{Computer Science}
\def\StudyBranch{Software and Data Engineering}

% An optional dedication: you can thank whomever you wish (your supervisor,
% consultant, a person who lent the software, etc.)
\def\Dedication{%
First of all, I would like to thank my supervisor Mgr. Rudolf Rosa, Ph.D. for all his time, guidance and valuable advices he gave me while working on this thesis. I would also like to thank my parents for their unlimited support and patience during my studies.
}

% Abstract (recommended length around 80-200 words; this is not a copy of your thesis assignment!)
\def\Abstract{%
This thesis deals with the problem of automatic generation of medical reports in the Czech langugage based on the input chest X-ray images using deep neural networks. The first part deals with the analysis of problem itself including comparison of existing solutions from several common points of view. In order to interpret medical images in the Czech language we present a fine-tuned a Czech GPT2 model specialized on medical texts based on the original pre-trained English GPT2 model along with its evaluation. In the second part the created Czech GPT2 is used for training neural network model for generating medical reports. The training was conducted on freely available data along with data pre-processing and their adjustment for the Czech language. Furthermore the model results are discussed and evaluated using standard metrics for natural language processing to determine the performance.
}

% 3 to 5 keywords (recommended), each enclosed in curly braces
\def\Keywords{%
{natural language processing}, {image captioning}, {x-ray}, {medical report generation}
}

%% The hyperref package for clickable links in PDF and also for storing
%% metadata to PDF (including the table of contents).
%% Most settings are pre-set by the pdfx package.
\hypersetup{unicode}
\hypersetup{breaklinks=true}

% Definitions of macros (see description inside)
\include{macros}

% Title page and various mandatory informational pages
\begin{document}
\include{title}

%%% A page with automatically generated table of contents of the master thesis

\tableofcontents

%%% Each chapter is kept in a separate file
\chapter*{Introduction}
\addcontentsline{toc}{chapter}{Introduction}

In hospital, inspecting the X-rays and writing corresponding medical reports is a hard work that requires experienced specialized doctors, of which there are not many. A great number of people visit hospitals daily and X-rays are taken for many of them. Automatic interpretation of X-ray images has a great potential to improve health care and it could be particularly helpful to doctors in order to distinguish serious cases from ordinary ones and overall accelerate and improve their work.\\

Automatic generation of radiology reports is a subset of a general problem called Image Captioning, i.e. generation of overall textual captions to input images. Image Captioning is a combination of Natural Language Processing and Computer Vision areas, experiencing a lot of progress in the last years. Most often the Image Captioning problem is solved using Deep Learning techniques. The specificity of this subset is that we do not want to generate just a general caption of the image, but the exact description of all findings contained in the given medical image. There were done multiple studies for this task in other languages but none in the Czech language.\\

Deep learning by its very nature has a wide range of uses in a medical sector as it can often capture complex relations in any kind of data with excellent performance results. Nevertheless, in the medical environment, the accuracy of predictions is crucial in order to determine the final diagnosis. Therefore, we should not consider the models as such as something that is unmistakably true but as an auxiliary tool that should help doctors to examine X-rays.\\

Inasmuch as it is not so challenging to detect fractures on the limbs, this area is less interesting than others which have a variety of diverse possible problems. One of these areas is the chest for which there exist multiple freely accessible datasets containing full textual medical reports. However, all these available datasets have one common downside, they are not in the Czech language. The natural question arises, where do we obtain these much needed data? We have to face and solve this core problem in our thesis.\\

\section*{Goals}
First of all, we will take a closer look at the problem itself. This includes breaking down the problem and analyzing all its parts individually together with presenting possible existing alternatives for each part. \\

The main target of our thesis is to train a neural network for the purpose of generating textual medical reports for X-ray images. An example of our problem can be seen in Figure \hyperref[fig01:ProblemExample]{1.1} for illustration.\\
\newpage

Our first goal is to fine-tune a language model directly for the Czech language. The language model will be specialized directly to medical texts in order to capture the essence of the problem. However, ahead of the medical specialization, we want to fine-tune a general Czech language model. Fine-tuning will be based on the original English GPT-2 model presented in \citet{radford2019language}.\\

Finally, we want to utilize our fine-tuned language models for training neural network models interpreting chest X-rays images and generating corresponding medical textual reports to them in the Czech language. This section also involves the overall data preparation directly for the Czech language. In addition, the training will be done in multiple setups. All possibilities will be evaluated with the purpose of determination of their final performance.\\

\section*{Thesis structure}

In the very first chapter we present a detailed description of our problem. Every aspect of our problem is introduced and all existing solutions or possibilities are discussed with their pros and cons. Moreover, we introduce there some of the important related works.\\

The following chapter is dealing with the design of the solution to our problem, with all reasonings and decisions made. This includes not only the final neural network model but also the language model fine-tuning and data preparation.\\

Technical details about the implemented scripts are described in the third chapter.\\

All experiments done with our models take their part in the fourth chapter, describing all used environments and different setups together with data variants. This section also contains partial results of work related to GPT-2 training.\\

The whole fifth chapter is then dedicated to an extensive evaluation of the experiments carried out in the preceding part.\\

Finally, in the epilog we discuss what we have accomplished in the thesis, what the resulting consequences are, and what the future possibilities are.
\chapter{Problem Analysis}
This chapter deals with the overall analysis of the problem itself. In the very beginning we present the definition of the problem. Every aspect of the problem is further discussed in detail along with a comparison of possible solutions. Moreover, the next section of the chapter describes data we work with and their alternatives. The final part of this chapter presents some of the important related works.

\section{Problem definition}

\section{Methods of generation}

\section{Data}
In previous part we talked about possible methods of generation. Another crucial aspect we need to discuss are data, which are a basic building block of our thesis. This part focuses on the analysis of the data we used in our thesis, but also on their alternatives. \\

In order to solve our task and train neural network we need to get dataset containing the X-rays images along with their textual descriptions and optionally some other attributes of the examined X-rays. Moreover, the fundamental feature we need is that the data must be in the Czech language.

\subsection{Existing datasets}
Medical environment provides a plenty of diverse potential problems, which can be researched. As already mentioned, in this thesis we focus specifically on the X-ray images. Because it is not so hard to detect fractures on the limbs, this area is not as interesting as others. One area that is rich in its diversity is the chest. As a result, this area is explored the most and therefore there exists multiple datasets with full textual mecidal reports. In the following section we describe some of them.

\subsubsection{OpenI}
OpenI dataset description
\subsubsection{MIMIC-CXR}
MIMIC-CXR dataset description

\subsection{Czech data}
All freely available datasets presented in the previous part have one common downside, namely they are not in the Czech language. As a part of elaboration of this thesis an intesive communication with real czech hospitals and other possible sources of real data took place. The goal of this communication was to create the very first open czech dataset of this kind. Processing of this kind of data would mean not only preparing the data into suitable format but also it would include proper anonymization of any personal information about the patients within the data. \\

However, inasmuch as the authentic patients data from hospitals are subject to strict privacy rules and we are not employees of any hospital, the institutions decided that they cannot provide the data in any way without the concious permission of patients given before the examination. With this result we need to find a different way how to obtain this much needed czech data.

\subsection{Translators}
In the previous sections we discovered that there is no dataset in the Czech language for our problem and there is no easy way how to get acces to the real data. The only thing left is to create a new artificial dataset using an automatic translation. We will compare different freely accesible translators and choose the right for our needs.

\subsubsection{DeepL}
DeepL description
\subsubsection{Google Translate}
GT description
\subsubsection{CUBBITT}
CUBBITT description

\section{Language models}
\subsection{GPT2}

\section{Related work}
The last section of this chapter is dedicated to description and comparison to some of the related works that solves identical or similar problem as we do.
\chapter{Design}
This chapter summarizes the overall design of our approach to generating textual medical reports for X-ray images. The problem consists of multiple independent parts we need to deal with. For each of them, we we will present the fundamentals of our solution, along with description of related problematics and decisions made.

\section{Our approach}
As we already mentioned in the Chapter \ref{sec:RelatedWork}, the overall solution for the final medical report generation model is based on the \citet{alfarghaly2021automated}. We have chosen this approach for multiple reasons. The main reasons to use this work as the backbone for our thesis are following:
\begin{enumerate}
	\item In the work the state-of-the-art GPT-2 model is utilized as the language model. This gave us a great opportunity as there was none Czech GPT-2 model available at time this thesis began.
	\item The encoder is already fine-tuned to extract visual features for specific dataset.
	\item All solution source code is freely accessible on the github\footnote[1]{\url{https://github.com/omar-mohamed/GPT2-Chest-X-Ray-Report-Generation}}.
\end{enumerate}

As in most works for image captioning, the architecture is encoder-decoder based with an attention mechanism. The high-level solution architecture is depicted in Figure \hyperref[fig01:OmarArchitecutre]{2.1}.

\begin{figure}[h]\centering
\includegraphics[width=145mm, height=57mm]{../img/OmarArchitecture}
\caption{Overall architecture used in our solution proposed in \citet{alfarghaly2021automated}.}
\label{fig01:OmarArchitecutre}
\end{figure}

\section{Czech GPT-2}
The aim of this work is to generate medical reports in the Czech language. In the previous parts, we decided to use the GPT-2 as the language model. However at the time of the beginning of this work, no Czech GPT-2 model was freely available and thus it was essentially necessary to create one. This section describes all the steps needed for fine-tuning the small English GPT-2 to the Czech language. Respectively, we train two versions of the Czech GPT-2 model. One trained on the general Czech textual data and one specialized specifically on Czech medical texts.

\subsection{Data}
In this section we will describe the possible data applicable for training of both the general and medical Czech GPT-2 model together with the decision made about the final data selection and data cleaning.

\subsubsection{General}
For the training of general Czech GPT-2 we have a plenty of data options we can choose from. However, there are important properties of the data we need to sastisfy. As we are transfer learning from English to Czech language we need to have sufficiently large data, so we ensure the GPT-2 will learn properly syntactical and semantical information. Moreover, the data have to be also heterogeneous enough, so the model can capture different types of information and not just for example newspaper articles from a specific area. In order to create a good enough general model we need to meet these criteria.\\

Several different datasets were investigated and tested for the training of the general Czech GPT-2 model.
\paragraph*{Czech Wikipedia} ~\\
\indent The first data we used for training the model is the Czech~Wikipedia~dump\footnote[2]{\url{https://dumps.wikimedia.org/cswiki/latest/cswiki-latest-pages-articles.xml.bz2}}. After extraction, the total size of the dataset is approximately 800 MB of raw text. The advantages of this dataset are its easy accessibility and fairly clean data quality. On the other hand, the data are very homogeneous despite the various topics. Each article is written in the general descriptive style. Moreover the data themself are not large enough, the trained model made many both syntactical and semantical mistakes during the text generation.

\paragraph*{Balanced Czech National Corpus} ~\\
\indent Another possibility was to use a balanced version of the Czech National Corpus\citep{11234/1-4635} as the original is composed mainly of journalistic articles. The balanced version tries to equalize the amount of data from each category. These categories include \textit{journalism}, \textit{poetry}, \textit{prose}, \textit{educational literature} etc. The major advantage of this dataset is its purity, the texts are syntacticaly correct without any undesirable non-Czech elements and written in the standard Czech language. The dataset does not have any significant downsides and the trained model understood Czech language without any significant ailments. In total, the dataset is comprised of 3,3 GB of raw text.

\paragraph*{OSCAR} ~\\
\indent OSCAR, from the \citet{ortiz-suarez-etal-2020-monolingual}, is a huge deduplicated multilingual corpus created from the Common~Crawl~corpus\footnote[3]{\url{https://commoncrawl.org/}} providing data for 166 different languages and available directly in the huggingface datasets library\footnote[4]{\url{https://huggingface.co/datasets/oscar}}. It consists of the text scraped from websites of very different kinds and thus the data are heteregeneous enough. Moreover, its huge size, as the czech part of the dataset occupies a total of 24 GB of raw text, is another major benefit. On the other hand, because the data are automatically scraped, they carry a noise in them. Besides that, not negligible part of the text are in the non-standard Czech language as the data come from diverse web sources such as forums etc. Nevertheless, the disadvantages are outweighed by the huge size of the corpus and together with following filtering of the text:
\begin{enumerate}
	\item We take only text that are at least 1200 characters longs as these texts tend to be longer articles written in the standard Czech language instead of advertisements, incomple texts etc.
	\item Any text containing control character are filtered out, because the text contains generally undesirable content.
\end{enumerate}

\paragraph*{Conclusion} ~\\
\indent We analyzed various datasets along with their overall properties. Furthermore, the advantages and disadvantages of each were discussed. As a result, we have chosen the \textbf{OSCAR} dataset due to its size and heterogeneity. The \textbf{Wikipedia} is too small and homogeneous. On the other hand the \textbf{Balanced Czech National Corpus} is heterogeneous enough, however it is an almost order of magnitude smaller than \textbf{OSCAR}.

\subsubsection{Medical}
Since we have a trained general Czech GPT-2 model from the previous section that already understands a Czech language, the final fine-tuning for medical environment does not require that much data. We need to specialize the model to understand the mecial environment inherently. For this purpose, we use a subset of the UFAL~Medical~Corpus~v. 1.0\footnote[5]{\url{https://ufal.mff.cuni.cz/ufal\_medical\_corpus}}. These data are further filtered to remove any inappropriate characters, lines and redundant structures. As a result, the data contain a total of 100 MB of raw medical texts. The texts comprise of general medical descriptions, articles and package leaflets for medicines.

\subsection{Training}
TODO - Popsat postup tréninku podle příspěvku(popis - obecně, trénování tokenizeru, postupné trénování) + learning rate finder(popis, obrázek) + diff lear. rates(popis - 4 skupiny) + ta trénovací křivka(popis - rozdíl oproti klasickýmu, obrázek), popsat ještě rozdíl mezi originállním přístupem a novým(gradual vs full)

\section{Medical dataset translation}
For machine learning in general and NLP tasks especially, the data quality is the alpha and omega of the performance of the final model. Since, as we have already mentioned in the Chapter \ref{sec:CzechData}, we do not have any Czech data directly for the medical examinations of X-rays, to obtain Czech data we must arrange ourselves in a different way. One potential way is to create a new artificial dataset using machine translation of existing datasets. This section discusses the required steps to build a quality dataset using translation.

\subsection{Translator choice}
In the previous part of the text, specifically in the Chapter \ref{sec:Translators}, we already discussed all possibilities for automatic translation. Our final choice for the translator is the CUBBITT as it provides REST API unlimited in the number of requests and volume.

\subsection{Preprocessing}
\label{sec:DataPreprocessing}
The most important part of our machine translation process is the preprocessing of the input text. We already outlined in the Chapter \ref{sec:datasets} that the data contain some noise in them as we are dealing with reports in natural language. Moreover, we will use CUBBITT translator, that doest not perfom auto-correction itself and cannot translate some patterns at all as we described in the Chapter \ref{sec:Cubbitt}. For these purposes we incorporate preprocessing before the translation as it would be beneficial to have all texts in standardized form in order to firstly, help CUBBITT with translation to get the report correctly translated, and secondly to ensure that our model receives and process all the data in a indentical report format.\\

Our preprocessing pipeline encompasses of following procedures. Some of them are dealing with general CUBBITT issues and other with specifics of the medical data.

\subsubsection*{Line starts}
First of all we start with a very simple procedure. We analyze all lines of the report and delete all white characters common for all lines on both ends. The purpose of this modification is to standardize the report format, while preserving its structure.

\subsubsection*{Anonymous sequences}
Inasmuch as the whole datasets are anonymized due to the legal reasons and privacy protection, the reports contain \qq{anonymous sequences}, such as \qq{XXXX} or \qq{\underline{{ }{ }{ }{ }{ }}}, denoting places with original private information about patients. However, these sequences can be attached to surrounding words forming undesirable words. As we already said, the CUBBITT does not auto-correct its input automatically, so these inputs will not be handled in any special way and therefore not translated. For this reason, we separate these sequences to form independent words.

\subsubsection*{Units}
Subsequent form of correction that we perform is the separation of numbers and units attached to them. In addition, this also includes general cases, where the number and subsequent word are glued together, while keeping specific medical terms with a similar structure. This is associated with the following step, as the units will not be true-cased properly without this procedure.

\subsubsection*{True-casing}
The most imporant part of the whole preprocessing pipeline is true-casing of the input text. This adjustment is necessary for two essential reasons. CUBBITT has problems with translation of any uppercase texts in general. Capturing the true-case of a text is a complex problem requiring either a large statistical language dictionary or a trained model in order to properly determine the case.\\

As medical reports are very specific area, the existing solutions for general text true-casing is inapplicable. Medical reports contain a lot of abbreviations and acronyms, which can be often confused with ordinary english words. Training the model for medical true-casing requires even more specific data for a certain domain, because in different contexts the common words can be treated differently. Moreover, obtaining flawless data to cover the entire specific domain is a challenging task. For these reasons, we chose the way of a statistical dictionary. Before the translation of the dataset begins, we create a statistical dictionary from the whole dataset containing the most often used form of every word. We also count with some exceptions, such as headings, that in some datasets can be in uppercase only.\\

Using the created dictionary, we deal with all uppercase words to assign them the proper form. The results of the true-casing preprocessing are demonstrated in the following examples:
\begin{itemize}
	\item (1a) \qq{EXAMINATION: CHEST (PORTABLE AP)} $\rightarrow$ \\ \phantom{(1a)} \qq{PŘEZKOUŠENÍ: CHEST (PORTABLE AP)}
	\item (1b) \qq{Examination: Chest (portable AP)} $\rightarrow$ \\ \phantom{(1b)} \qq{Vyšetření: Hrudník (přenosný~AP)}
	\item (2a) \qq{SMALL RIGHT PLEURAL ABNORMALITY} $\rightarrow$ \\ \phantom{(2a)} \qq{MALÉ PRÁVO PLEURÁLNÍ ABNORMALITY}
	\item (2b) \qq{Small right pleural abnormality} $\rightarrow$ \\ \phantom{(2b)} \qq{Malá pravá pleurální abnormalita}
\end{itemize}

\subsubsection*{Paragraphs structure}
In some of the medical reports the section headings and corresponding texts do not begin on the same lines. We adjust these situations to a form where each heading and the text belonging to it always starts on the same line, for two reasons. Firstly, we want to normalize the report structure in general and secondly, we want to move the section content as close as possible to the heading, so the translator and even the final model have the context close to each other.

\subsubsection*{Capitalization}
Another part of the preprocessing pipeline is a simple capitalization of each heading and each sentence in the report. This text capitalization process helps CUBBITT not only in the case of medical data, but in general during the translation process of some texts to better understand the boundaries between sentences.

\subsubsection*{Time}
One of the patterns that CUBBITT doest not recognize nor auto-corrects itself in general are times. This applies both, to the specification of hours and minutes, and to the  part of the day specification. If any of these parts are incorrectly formatted, the time will we translated incorrectly or not translated at all, and thus we would lose some information or it could damage the fluency of the translated text. For these reasons, we apply preprocessing to normalize all times. We can see the difference in the following examples:
\begin{itemize}
	\item \qq{Chest radiograph at 1045PM} $\rightarrow$ \qq{Rentgen hrudníku v 1045PM}
	\item \qq{Chest radiograph at 10:45 PM} $\rightarrow$ \qq{Rentgen hrudníku ve 22:45}
\end{itemize}

\subsubsection*{White spaces}
After all previous procedures, we apply one last very simple final modification, namely, we squash all the whitespace characters inside each line into a single space, while maintaining the format from the very first preprocessing step described above. This step is performed only for normalization purposes.

\subsubsection*{Lowercasing}
The last preprocessing procedure we will mention is lowercasing of the whole text followed by capitalizing the first word of each sentence. This is a separate procedure that was used in the earlier phases of elaboration of this work as CUBBITT has problem with uppercased word as we already mentioned.
















\chapter{Experiments}
Úvod

\section{Environment}
Popsat prostředí na kterém jsme trénovali - it4i cluster a aic cluster

\section{Czech GPT-2}
Popis toho tréninku - odkaz na předchozí kapitolu.
Výsledky z trénování
* Normal Czech GPT-2
* Medical Czech GPT-2
Ukázky výsledků z obou
Popis scriptů

\section{Dataset translation}
Popsat, jak vypadá ten script, jak funguje a že jsme nakonec překládali jen open-i
+ Výsledky

\section{Medical report generation model}
Popsat, že jsme použili data z Open-I jako v původním článku, jelikož nám jde hlavně o proof-of-concept + Mimic nemá stejné tagy -> muselo by se přetrénovat celý CNN backbone.
+ Výsledky
+ Popis toho, jak jsme upravili data do stejného formátu jako v článku, popis těch skriptů je na githubu toho článku
\chapter{Evaluation}

\section{Experiments}

\section{Automatic evaluation}
\subsection{Metrics}
\subsection{Results}

\section{Manual evaluation}
\subsection{Method}
\subsection{Results}

\section{Examples}

\chapter*{Conclusion}
\addcontentsline{toc}{chapter}{Conclusion}

In this thesis, we have been dealing with a problem of automatic generation of medical reports from chest X-rays in the Czech language. Many studies have been conducted for English, but none for Czech, to the best of our knowledge. The first part has presented a detailed analysis of the overall problem. We have introduced neural network architectures and data alternatives along with their properties that are commonly used for this problem and a survey of related works.\\

We have designed our approach based on the \citet{alfarghaly2021automated} paper. Nevertheless, as the original solution is intended for English, we have gradually solved all associated problems. Firstly, we have described a fine-tuning process for the two Czech GPT-2 models on general Czech and medical data with reasoning about different possibilities of data. Furhermore, since there are no Czech medical datasets comprising of X-ray images and their corresponding reports, we proposed a solution for an automatic translation of English datasets. \\

We have then described the implementation details and all important source codes of our solution separately for each independent part.\\

In the next part, the final experiments took their place. We have performed two GPT-2 model trainings according to our design along and presented their results. Next, we have translated the Indianta University chest X-ray dataset into the Czech language for our final task. Afterwards, we have conducted experiments for the medical report generation in multiple different setups of the neural network. Individual setups differed in the used language model or the extent to which the network has been trained.\\

Finally, all ran experiments have been evaluated both automatically and manually. The automatic evaluation consisted of numerous stadard NLP metrics and it have given us an indication of which models could be performing good. These assumptions have been further verified by manual evaluation by a trained physician. We found out that the medical report generation for the Czech language is possible and has future potential. Moreover, fine-tuning the Czech GPT-2 model on medical texts helps to improve the outputs accuracy.
\newpage

\section*{Future work}
There are several aspects that could be further worked on in the future. The Czech GPT-2 models could be improved by training on cleaner and more extensive data with data preparation so that within one sequence there is always only one source of text. Moreover, instead of fine-tuning from the original English model to Czech, we could use some trained Slavic GPT-2 model as our base in order to get better results and more coherent texts.
For the final chest X-ray report generation, we could modify the model or even create our own in order to train on a larger dataset. In addition, our model could also use newer architectures to improve performance. Furthermore, a translator designed specifically for the translation of medical texts would greatly improve the quality of the translations. Another major improvement would be to obtain real Czech hospital data without translation noise on which we could train. Based on the data we would have, we might even train additional objectives, such as bounding boxes to focus on relevant parts of the X-ray image.

%%% Bibliography
\include{bibliography}

%%% Figures used in the thesis (consider if this is needed)
\listoffigures

%%% Tables used in the thesis (consider if this is needed)
%%% In mathematical theses, it could be better to move the list of tables to the beginning of the thesis.
\listoftables

%%% Abbreviations used in the thesis, if any, including their explanation
%%% In mathematical theses, it could be better to move the list of abbreviations to the beginning of the thesis.
\chapwithtoc{List of Abbreviations}

%%% Attachments to the master thesis, if any. Each attachment must be
%%% referred to at least once from the text of the thesis. Attachments
%%% are numbered.
%%%
%%% The printed version should preferably contain attachments, which can be
%%% read (additional tables and charts, supplementary text, examples of
%%% program output, etc.). The electronic version is more suited for attachments
%%% which will likely be used in an electronic form rather than read (program
%%% source code, data files, interactive charts, etc.). Electronic attachments
%%% should be uploaded to SIS and optionally also included in the thesis on a~CD/DVD.
%%% Allowed file formats are specified in provision of the rector no. 72/2017.
\appendix
\chapter{Attachments}

\section{First Attachment}

\openright
\end{document}
